% Created 2023-03-17 Fri 15:55
% Intended LaTeX compiler: pdflatex
\documentclass[11pt]{article}
\usepackage[utf8]{inputenc}
\usepackage[T1]{fontenc}
\usepackage{graphicx}
\usepackage{longtable}
\usepackage{wrapfig}
\usepackage{rotating}
\usepackage[normalem]{ulem}
\usepackage{amsmath}
\usepackage{amssymb}
\usepackage{capt-of}
\usepackage{hyperref}
\author{bruno cuconato}
\date{\today}
\title{}
\hypersetup{
 pdfauthor={bruno cuconato},
 pdftitle={},
 pdfkeywords={},
 pdfsubject={},
 pdfcreator={Emacs 28.2 (Org mode 9.5.5)},
 pdflang={English}}
\begin{document}
We define Points as an opaque type, they are not explicitly described,
and we can't ‘look inside’ them.

With the exception of Points, in Lean we represent the geometric
objects by their constructions (or deconstructions, depending on how
you look at it).

A pair of points form a Segment; in Lean, this is a structure
containing the two points, together with a proof that they are
different from other (i.e., not the same point). A PolySegment is
isomorphic to a list of Segments, having a constructor for the empty
PolySegment, a constructor that raises a Segment into a singleton
PolySegment, and one that joins two PolySegments into one.

A Face is a pair of PolySegments and a proof that they form a Jordan
curve. A PolyFace is isomorphic to a list of Faces, having a
constructor for the empty PolyFace, a constructor that raises a Face
into a singleton PolyFace, and one that joins two PolyFaces into one.

A Volume is a pair of PolyFaces together with a proof that they form a
closed volume. A PolyVolume is isomorphic to a list of Volumes, having
a constructor for the empty PolyVolume, a constructor that raises a
Volume into a singleton PolyVolume, and one that joins two PolyVolumes
into one.

A TruncationOf an object is contructed recursively. The base case is
that for any non-empty version of the object, the empty version is a
TruncationOf it. So for the case of PolyVolumes, for any non-empty
PolyVolume the empty PolyVolume is a TruncationOf it. Then, given two
objects, the first of which is TruncationOf the second one, we have
that the join of the first object with any third object is a
TruncationOf the second one with the same third object (provided we
can perform both joins, that is, their results are well-formed
geometric objects).

Given two PolyVolumes \(q\) and \(p\) and a constructive proof that the
\(q\) is a TruncationOf \(p\), we can derive Zolt's theorem by induction
on the TruncationOf construction: in the base case, we have that \(q\)
is the empty PolyVolume \(\varepsilon\), and so we use the
\(\varepsilon_2\) rule to show that \(\varepsilon < p\). In the inductive
case we have that \(p\) and \(q\) are actually \(p';r\) and \(q;r\), and we
have a proof of \(q' < p'\). With this proof and a trivial proof that \(r
\leq r\) we can invoke the \(lt_1\) rule to show that \(q';r < p';r\;_\square\)

\end{document}
%%% Local Variables:
%%% mode: latex
%%% TeX-master: t
%%% End:
