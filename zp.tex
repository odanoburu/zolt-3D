\documentclass[12pt]{article} %

\usepackage{fontspec}
% switch to a monospace font supporting more Unicode characters
\setmonofont{FreeMono} %

\usepackage[lmargin=1.8in,rmargin=1.8in]{geometry} % See geometry.pdf to learn the layout options. There are lots.
\geometry{a4paper} % or letterpaper (US) or a5paper or....

\usepackage{setspace}
\onehalfspacing{} % 130% spacing between lines: %
\frenchspacing{}

\usepackage{amsmath} %
\usepackage{amssymb} %

% https://leanprover.github.io/lean4/doc/syntax_highlight_in_latex.html
\usepackage{minted}
% instruct minted to use our local theorem.py
\newmintinline[lean]{lean4.py:Lean4Lexer -x}{bgcolor=white}
\newminted[leancode]{lean4.py:Lean4Lexer -x}{fontsize=\footnotesize}
\usemintedstyle{tango}  % a nice, colorful theme

\usepackage{newunicodechar} %
\newunicodechar{ₚ}{\ensuremath{_{p}}} %

\newcommand{\leanline}[1]{\mintinline{lean}{#1}} %
\newcommand{\point}{\mintinline{lean}{Point}} %
\newcommand{\segment}{\mintinline{lean}{Segment}} %
\newcommand{\psegment}{\mintinline{lean}{PolySegment}} %
\newcommand{\face}{\mintinline{lean}{Face}} %
\newcommand{\pface}{\mintinline{lean}{PolyFace}} %
\newcommand{\volume}{\mintinline{lean}{Volume}} %
\newcommand{\pvolume}{\mintinline{lean}{PolyVolume}} %
\newcommand{\truncof}{\mintinline{lean}{TruncationOf}} %

\usepackage{cleveref} %

\newcommand{\zp}{\ensuremath{Z_{p}}}

\date{\today}
\title{de Zolt's theorem}

\begin{document}
\maketitle %

\section{Geometric objects in Lean}\label{sec:geom-objects-lean} %

We take geometric points as given, so we define them as an opaque Lean type (henceforth written as \point). %
This means they are not explicitly described, and we can't ‘look inside’ them. %

\begin{leancode}
structure Point : Type
\end{leancode}
The treatment given to \point{} is exceptional: all the other geometric objects are represented in Lean by their constructions (or deconstructions, depending on how you look at it). %

\begin{leancode}
structure Segment : Type where
  p1 : Point
  p2 : Point
  neq : p1 ≠ p2
\end{leancode}

A pair of points form a \segment; in Lean, this is a structure
containing the two points, together with a proof that they are
different from each other (i.e., not the same point). %

The type of \psegment{} is isomorphic to a list of \segment, having a constructor for the empty \psegment, a constructor that raises a \segment{} into a singleton \psegment, and one that joins two values of \psegment{} into one. %
\begin{leancode}
inductive PolySegment : Type where
| s₀ : PolySegment
| s₁ : Segment → PolySegment
| s₂ : PolySegment → PolySegment → PolySegment
\end{leancode}

A \face{} is a pair of \psegment{} and a proof that they form a Jordan
curve. %
For this purpose, a Jordan curve is defined opaquely as a predicate \leanline{IsJordan} over a pair of \psegment{}. %
A \pface{} is isomorphic to a list of \face, having a constructor for the empty \pface, a constructor that raises a \face{} into a singleton \pface, and one that joins two values of type \pface{} into one. %
\begin{leancode}
structure Face : Type where
  s1 : PolySegment
  s2 : PolySegment
  jordan : IsJordan s1 s2

inductive PolyFace : Type where
| f₀ : PolyFace
| f₁ : Face → PolyFace
| f₂ : PolyFace → PolyFace → PolyFace
\end{leancode}

A \volume{} is a pair of \pface{} together with a proof that they form a
closed volume. %
For this, a closed volume is defined opaquely as a predicate \leanline{IsClosed} over a pair of \pface. %
A \pvolume{} is isomorphic to a list of \volume, having a constructor for the empty \pvolume, a constructor that raises a \volume{} into a singleton \pvolume, and one that joins two values of type \pvolume{} into one. %
\begin{leancode}
structure Volume : Type where
  vol1 : PolyFace
  vol2 : PolyFace
  closed : IsClosed vol1 vol2

inductive PolyVolume where
| v₀ : PolyVolume
| v₁ : Volume → PolyVolume
| v₂ : PolyVolume → PolyVolume → PolyVolume
\end{leancode}

\section{The \zp{} system}\label{sec:zp-system} %

We can classify the rules of the \zp{} system in two groups: the ones related to the construction of geometrical objects, and the ones related to comparing them for size. %

The first rules are implemented in Lean as typeclass, of which the types described in \cref{sec:geom-objects-lean} are instances. %
\begin{leancode}
class Zₚ (a : Type u) where
  ε : a
  cmp : a → a → Prop
  join : (p : a) → (q : a) → cmp p q → a
  NonEmpty : a → Prop
\end{leancode}

In Lean, typeclasses are used as a way to overload notation. %
Using this case as an example, we may use \(\varepsilon\) to denote the empty geometric object of any type which is an instance of \zp{}, as we do in the deductive rules for \zp{}. %

The attentive reader will have noticed that in \cref{sec:geom-objects-lean} we defined constructors such as \leanline{PolyVolume.v₂} that may build invalid (not well-formed) geometric objects. %
For a \pvolume{} to be well-formed when constructed from a pair of \pvolume, we need them to have an intersection forming a \face. %
To enforce such well-formedness restrictions, the \zp{} typeclass defines the \leanline{cmp} predicate, that is used as an argument to the \leanline{join} function. %

Further on, we we will also need a predicate establishing that the geometric object is not the empty one, so we add it here. %

To illustration how this works in practice, below is the instantiation of \pvolume{} as part of the \zp{} typeclass: %
\begin{leancode}
opaque PolyVolume.HasFaceIntersection
  : PolyVolume → PolyVolume → Prop

def PolyVolume.NonEmpty : PolyVolume → Prop
| v₀ => False
| v₁ _ => True
| v₂ _ _ => True

instance : Zₚ PolyVolume where
  ε := PolyVolume.v₀
  cmp := PolyVolume.HasFaceIntersection
  join := λ p, v, _cmp => PolyVolume.v₂ p v
  NonEmpty := PolyVolume.NonEmpty
\end{leancode}
As one can see, we need the auxiliary definition of the \leanline{HasFaceIntersection} predicate over a pair of \pvolume{} to be used as our \leanline{cmp} predicate for \pvolume. %
To join two \pvolume{}s in one, we use the regular \leanline{v₂}, but the (otherwise unused) \leanline{_cmp} argument guarantees that the result is well-formed. %
We also define the \leanline{NonEmpty} predicate for \pvolume{} in the obvious way. %

The instantiations for the other geometrical objects is similarly made (with the appropriate predicates), but they are not necessary for the proof of the de Zolt theorem in \zp. %

Now we need to define the rules for comparing \zp{} values. %
\(<\) and \(\leq\) are defined inductively over pairs of \zp{} values: %
\begin{leancode}
mutual
  variable {t} [Zₚ t]

  inductive Zₚ.le : t → t → Prop where
  | refl {p : t} : le p p
  | le₁ : ∀ {p₁ q₁ p₂ q₂ : t}, le p₁ q₁ → le p₂ q₂
        → (pc : cmp p₁ p₂) → (qc : cmp q₁ q₂)
        → le (join p₁ p₂ pc) (join q₁ q₂ qc)

  inductive Zₚ.lt : t → t → Prop
  | ε₁ : ∀ {p q : t}, (pqc : cmp p q) → lt p (join p q pqc)
  | ε₂ : ∀ {p q : t}, (pqc : cmp p q) → lt q (join p q pqc)
  | lt₁ : ∀ {p₁ q₁ p₂ q₂ : t}, lt p₁ q₁ → le p₂ q₂
        → (pc : cmp p₁ p₂) → (qc : cmp q₁ q₂)
        → lt (join p₁ p₂ pc) (join q₁ q₂ qc)
  | lt₂ : ∀ {p₁ q₁ p₂ q₂ : t}, le p₁ q₁ → lt p₂ q₂
        → (pc : cmp p₁ p₂) → (qc : cmp q₁ q₂)
        → lt (join p₁ p₂ pc) (join q₁ q₂ qc)
end
\end{leancode}
The constructors of the \leanline{le} (for \(\leq\)) and \leanline{lt} (for \(<\)) types correspond to the \zp{} deductive rules of the same name. %

\section{The proof}\label{sec:de-zolt-theorem} %

The final piece we need to prove the de Zolt theorem is the definition of a truncation of a geometric object (in Lean this means any value of a type which is an instance of the \zp{} typeclass). %
We define truncation inductively in the following Lean code: %
\begin{leancode}
section Truncation
variable {t} [Zₚ t]

inductive Zₚ.TruncationOf : t → t → Prop where
| t₀ {p : t} : NonEmpty p → TruncationOf ε p
| t₁ {p q r : t} : (pr : cmp p r) → (qr : cmp q r)
    → TruncationOf p q
    → TruncationOf (join p r pr) (join q r qr)

end Truncation
\end{leancode}

A \truncof{} value is thus contructed recursively. %
The base case is that for any non-empty version of a geometric object, the empty object is a \truncof{} it. %
So for the case of a \pvolume, for any non-empty \pvolume{} the empty \pvolume{} is a \truncof{} it. %
For the inductive case, given three geometric objects, the first of which is a \truncof{} the second one, we have that the join of the first object with the third object is a \truncof{} the second one with the same third object. %
This is only true provided we can perform both of these joins, that is, that their results are well-formed geometric objects; this is guaranteed by the two \leanline{cmp} arguments. %

We are finally ready for the statement of the de Zolt theorem: %
\begin{leancode}
theorem PolyVolume.zolt {q p : PolyVolume}
  (isTrunc : Zₚ.TruncationOf q p)
  : Zₚ.lt q p
\end{leancode}
That is, if \(p, q\) are \pvolume, and \(q\) is a \truncof{} \(p\), we have that \(q < p\). %

The proof is by induction on the \truncof{} construction: in the base case, we have that \(q\) is the empty \pvolume{} \(\varepsilon\), and so we use the
\(\varepsilon_2\) rule to show that \(\varepsilon < p\). %
In the inductive case we have that \(p\) and \(q\) are actually \(p';r\) (\leanline{join p' r}) and \(q;r\) (\leanline{join q r}), and we
have a proof of \(q' < p'\). %
With this proof and the trivial proof of \(r \leq r\) we can invoke the \(lt_1\) rule to show that \(q';r < p';r\;_\square\) %

Or, in Lean: %

\begin{leancode}
theorem PolyVolume.zolt {q p : PolyVolume}
  (isTrunc : Zₚ.TruncationOf q p)
  : Zₚ.lt q p :=
  match isTrunc with
  | Zₚ.TruncationOf.t₀ _ =>
      have pεcmp : Zₚ.cmp p v₀
        := HasFaceIntersection_comm EmptyAlwaysHasFaceIntersection
      (Eq.subst Zₚ.empty_right_join <| Zₚ.lt.ε₂ pεcmp)
  | Zₚ.TruncationOf.t₁ (p := w) (q := u) (r := r) wrcmp urcmp qIsTruncOfp =>
    have w_lt_u : Zₚ.lt w u := zolt qIsTruncOfp
    have r_le_r : Zₚ.le r r := Zₚ.le.refl
    Zₚ.lt.lt₁ w_lt_u r_le_r wrcmp urcmp
\end{leancode}

\end{document}
%%% Local Variables:
%%% mode: latex
%%% TeX-master: t
%%% TeX-engine: xetex
%%% TeX-command-extra-options: "-shell-escape"
%%% fill-column: 100000
%%% eval: (visual-line-mode 1)
%%% End:
